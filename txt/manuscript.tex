%% 
%% Copyright 2007, 2008, 2009 Elsevier Ltd
%% 
%% This file is part of the 'Elsarticle Bundle'.
%% ---------------------------------------------
%% 
%% It may be distributed under the conditions of the LaTeX Project Public
%% License, either version 1.2 of this license or (at your option) any
%% later version.  The latest version of this license is in
%%    http://www.latex-project.org/lppl.txt
%% and version 1.2 or later is part of all distributions of LaTeX
%% version 1999/12/01 or later.
%% 
%% The list of all files belonging to the 'Elsarticle Bundle' is
%% given in the file `manifest.txt'.
%% 

%% Template article for Elsevier's document class `elsarticle'
%% with numbered style bibliographic references
%% SP 2008/03/01

\documentclass[preprint,12pt, a4paper]{elsarticle}

%% Use the option review to obtain double line spacing
%% \documentclass[authoryear,preprint,review,12pt]{elsarticle}

%% For including figures, graphicx.sty has been loaded in
%% elsarticle.cls. If you prefer to use the old commands
%% please give \usepackage{epsfig}

%% The amssymb package provides various useful mathematical symbols
\usepackage{amssymb}
%% The amsthm package provides extended theorem environments
\usepackage{amsthm}
\usepackage{mathtools}
%% The lineno packages adds line numbers. Start line numbering with
%% \begin{linenumbers}, end it with \end{linenumbers}. Or switch it on
%% for the whole article with \linenumbers.
\usepackage{lineno}

\usepackage{float}

\usepackage{todonotes} 


\restylefloat{table}

\newcommand{\eg}{{\emph{e.g.\/}}}
\newcommand{\ie}{{\emph{i.e.\/}}}
\newcommand{\ket}[1]{\ensuremath{|#1\rangle}}
\newcommand{\bra}[1]{\ensuremath{\langle#1|}}
\newcommand{\ketbra}[2]{\ensuremath{\ket{#1}\bra{#2}}}
\newcommand{\proj}[1]{\ensuremath{\ketbra{#1}{#1}}}
\newcommand{\braket}[2]{\ensuremath{\langle{#1}|{#2}\rangle}}
\newcommand{\floor}[1]{\ensuremath{\lfloor #1 \rfloor}}
\newcommand{\complexity}[1]{\ensuremath{\mathbf{#1}}}
\newcommand{\new}[1]{ \textcolor{red}{#1} }
\newcommand{\1}{{\rm 1\hspace{-0.9mm}l}}
\newcommand{\Id}{{\rm 1\hspace{-0.9mm}l}}
\newcommand{\connected}{\sim}
\newcommand{\SPAN}{\mathrm{span}}
\newcommand{\Lrm}{\ensuremath{\mathrm{L}}}
\newcommand{\Urm}{\ensuremath{\mathrm{U}}}
\newcommand{\ee}{\ensuremath{\mathrm{e}}}
\newcommand{\dd}{\ensuremath{\mathrm{d}}}
\newcommand{\ii}{\ensuremath{\mathrm{i}}}
\newcommand{\EE}{\mathcal{E}}
\newcommand{\XX}{\mathcal{X}}
\newcommand{\MM}{\mathcal{M}}
\newcommand{\NN}{\mathcal{N}}
\newcommand{\DD}{\mathcal{D}}
\newcommand{\TT}{\mathcal{T}}
\newcommand{\PP}{\mathcal{P}}
\renewcommand{\SS}{\mathcal{S}}
\newcommand{\UU}{\mathcal{U}}
\newcommand{\HH}{\mathcal{H}}
\newcommand{\DU}{\mathcal{DU}}
\newcommand{\NOT}{\sigma_x}
\newcommand{\idop}[1][\XX]{\ensuremath{\1_{#1}}}
\newcommand{\diaguni}{\ensuremath{\mathcal{DU}}}
\newcommand{\diag}{\mathrm{diag}}
\newcommand{\tr}{\mathrm{tr}}
%\DeclareMathOperator{\diag}{diag}
%\DeclareMathOperator{\diag}{diag}
\journal{SoftwareX}


\usepackage{amsmath}
\newtheorem{theorem}{Theorem}
\newtheorem{proposition}{Proposition}
\newtheorem{remark}{Remark}
\newtheorem{scheme}{Scheme}
\newtheorem{lemma}{Lemma}

\begin{document}

\begin{frontmatter}

%% Title, authors and addresses

%% use the tnoteref command within \title for footnotes;
%% use the tnotetext command for theassociated footnote;
%% use the fnref command within \author or \address for footnotes;
%% use the fntext command for theassociated footnote;
%% use the corref command within \author for corresponding author footnotes;
%% use the cortext command for theassociated footnote;
%% use the ead command for the email address,
%% and the form \ead[url] for the home page:
%% \title{Title\tnoteref{label1}}
%% \tnotetext[label1]{}
%% \author{Name\corref{cor1}\fnref{label2}}
%% \ead{email address}
%% \ead[url]{home page}
%% \fntext[label2]{}
%% \cortext[cor1]{}
%% \address{Address\fnref{label3}}
%% \fntext[label3]{}

\title{Benchmarking on Rigetti architecture}

%% use optional labels to link authors explicitly to addresses:
%% \author[label1,label2]{}
%% \address[label1]{}
%% \address[label2]{}

\author{Paulina Lewandowska}
\author{\L ukasz Pawela}

\address{Institute of Theoretical and Applied Informatics, Polish Academy
	of Sciences, Ba{\l}tycka 5, 44-100 Gliwice, Poland}

\begin{abstract}
We introduce library ?, a comprehensive open source framework for benchmarking 
Amazon Braket Rigetti IonQ architectures ? 
The framework is built around  the scheme of discrimination of von Neumann 
measurements. The library provides algorithm for ... Our main aim is creating a 
common platform  for open research.  ? 


\end{abstract}

\begin{keyword}
%% keywords here, in the form: keyword \sep keyword
Benchmarking of Rigetti architecture \sep Discrimination of quantum 
measurements \sep Von Neumann measurements discrimination 

%% PACS codes here, in the form: \PACS code \sep code

%% MSC codes here, in the form: \MSC code \sep code
%% or \MSC[2008] code \sep code (2000 is the default)

\end{keyword}

\end{frontmatter}

\section*{Required Metadata}
\label{}

\section*{Current code version}
\label{}

Ancillary data table required for subversion of the codebase. Kindly replace 
examples in right column with the correct information about your current code, 
and leave the left column as it is.

\begin{table}[H]
\begin{tabular}{|l|p{6.5cm}|p{6.5cm}|}
\hline
\textbf{Nr.} & \textbf{Code metadata description} & \textbf{Please fill in this 
column} \\
\hline
C1 & Current code version & For example v42 \\
\hline
C2 & Permanent link to code/repository used for this code version & For 
example: $https://github.com/mozart/mozart2$ \\
\hline
C3 & Code Ocean compute capsule & For example: 
$https://codeocean.com/2017/07/30/neurospeech-colon-an-open-source-software-for-parkinson-apos-s-speech-analysis/code$\\
\hline
C4 & Legal Code License   & List one of the approved licenses \\
\hline
C5 & Code versioning system used & For example svn, git, mercurial, etc. put 
none if none \\
\hline
C6 & Software code languages, tools, and services used & For example C++, 
python, r, MPI, OpenCL, etc. \\
\hline
C7 & Compilation requirements, operating environments \& dependencies & \\
\hline
C8 & If available Link to developer documentation/manual & For example: 
$http://mozart.github.io/documentation/$ \\
\hline
C9 & Support email for questions & \\
\hline
\end{tabular}
\caption{Code metadata (mandatory)}
\label{} 
\end{table}


\linenumbers

%% main text

The permanent link to code/repository or the zip archive should include the 
following requirements: 

README.txt and LICENSE.txt.

Source code in a src/ directory, not the root of the repository.

Tag corresponding with the version of the software that is reviewed.

Documentation in the repository in a docs/ directory, and/or READMEs, as 
appropriate.




\section{Motivation and significance}

Noisy intermediate-scale quantum
(NISQ)~\cite{preskill} devices are currently storming the market. We
have a selection of readily available quantum computers based on different
architectures. Historically, the first widely introduced quantum computing
architecture is D-Wave's quantum annealer~\cite{}. 

https://quantumcomputingreport.com/scorecards/qubit-technology/

Additional approaches to quantum computing are currently in development. For 
instance we have Ionq's computer based on ion traps and Xanadu's optical 
lattices. As they are in prototype versions no access to general public is 
provided. 

Next, we have computers implementing the gate model of quantum computation.
These are the best potentially developed machines, which can be thought
of as fully quantum computers, by which we understand that the qubits can be in
an entangled state. 

Currently one of the main providers of such architectures is Rigetti and its 
Quantum Cloud Services platform. The company has
released three generations of QPUs so far: Rigetti 8Q Agave, Rigetti 19Q Acorn
and Rigetti 16Q Aspen-1. The most recent one was deployed in November 2018 and
is a downgrade in terms of number of qubits compared to the previous 19 qubit
one. However, the new chip is reportedly much more robust to noise and the
proposed architecture is much easier to scale up. The new unit is supposed to
serve as a building block for a 128 qubit
system~\footnote{https://medium.com/rigetti/the-rigetti-128-qubit-chip-and-what-it-means-for-quantum-df757d1b71ea}.
As was the case for IBM, Rigetti also provides a \texttt{Python} library, 
\texttt{pyquil}~\cite{}, which enables easy access to the machine.

\todo[inline]{co napisac o tym Amazonie, czy nic ?}

\todo[inline]{rysunki architektur}

%
%\begin{figure}[h]
%	\centering 
%	
%	\includegraphics[width=0.9\linewidth]{aspen.png} 
%	\caption{?}
%	\label{fig:aspen}
%\end{figure}


Now the natural question arises: how to construct a good metric of the
computational power of such devices? 
Lately, there have emerged propositions from the scientific community on how to
benchmark such devices.

 First let us mention the work by Michielsen
\emph{et.al.}~\cite{michielsen2017benchmarking}. In there the authors study the
IBM-QE device. Their first test is the creation of the entangled state
$\frac{1}{\sqrt{2}} (\ket{01} - \ket{10})$. Next they implement a
two-qubit+two-qubit adder. Another test is an identity operation, realized by an
even number of CNOT operations. Their final test is a quantum error correction
scheme (is it possible with measurement-controlled-operations?). The final
conclusion is that, except for simple circuits, the device fails to return
correct results.

Another, recent approach is utilizing quantum communication protocols in testing
of quantum architectures~\cite{zhukov2019quantum}. Here, the authors decide to
utilize superdense coding and the BB84 quantum cryptography protocol as
benchmarks for the IBM Q devices. They utilize the mutual information of the
transferred bits as a figure of merit for the quantum computer.

Other approaches focus on generative model training. One of these
works~\cite{hamilton2018generative} is aimed at small architectures, up to 5
qubits. Another recently presented is much more
robust~\cite{benedetti2018generative}. The main drawback of both of them is the
level of complication.



The goal of this work is to introduce a new benchmark for Rigetti 
architecture...


\label{}
\todo[inline]{
	Introduce the scientific background and the motivation for developing the 
	software. - motywacje 
	
	Explain why the software is important, and describe the exact (scientific) 
	problem(s) it solves. - czemu ejst wazne, jakie problemy rozwiazuje 
	
	Indicate in what way the software has contributed (or how it will 
	contribute in the future) to the process of scientific discovery; if 
	available, this is to be supported by citing a research paper using the 
	software. - do czego sie to przyczyni w przyszlosci 
	
	Provide a description of the experimental setting (how does the user use 
	the software?). - w jaki sposób użytkownik korzysta z oprogramowania?
	
	Introduce related work in literature (cite or list algorithms used, other 
	software etc.). - inne oprogramowania i algorytmy }



\section{Software description}
\label{}
\todo[inline]{Describe the software in as much as is necessary to establish a 
vocabulary needed to explain its impact. }

We will first give a general overview of the structure of the code in 
Section~\ref{sec:sortware-architecture} and then provide additional details on 
the functionality of benchmark's schemes  in 
Section~\ref{sec:sortware-functionalities}.

\subsection{Software Architecture}\label{sec:sortware-architecture}
\label{}
\todo[inline]{
	Give a short overview of the overall software architecture; provide a 
	pictorial component overview or similar (if possible). If necessary provide 
	implementation details.}


\subsection{Software Functionalities}\label{sec:sortware-functionalities}
\todo[inline]{Present the major functionalities of the software.}

In the scope of this Paper, we aim at  exploring ideas revolving around quantum 
gate model-inspired computing devices, and assess their feasibility to 
benchmark a modern quantum architectures. The main idea that we have envisioned 
for this project is  introducing new concepts of algorithms for NISQ devices 
benchmarking. This project aims to characterize the computing power and 
investigate possible practical applications of such devices having access by 
Amazon Braket.  First of all, we will focus on constraining quantum algorithms 
being relatively easy to implement on quantum architectures with 
straightforward, intuitive operational interpretation. Moreover, we need to 
possess a scalability, that is the ability to adjust to large range of 
architectures size.  
These properties will give us major advantage over the existing approaches  of 
benchmarking. 


This idea is built upon two specific tasks that are to be achieved within this 
proposal. 
The first task is to   desing a benchmark based on discrimination of von 
Nemaumm measurements scheme. While the second one is to extension  this 
approach to benchmark on the grounds of certification of von Neumann 
measurements scheme.




\subsubsection{Model specification}
Theoretically, scenario of this set up assume that Alice and Bob have an 
unknown measurement device, black--box.  Alice has hiden one of the two 
measurements  in the black box.  The only information which Bob has is that it 
performs one of such  measurements. 
The  goal of discrimination of von Neumann measurements  is to determine by Bob 
whether it is possible to discriminate measurements perfectly, i.e. with 
probability equals one. If this is not possible, he  would like to know the 
upper bound of such a probability.  Theoretical results of discrimination of 
von Neumann measurements is well-known and characterized 
in~\cite{puchala2018strategies}.    The scheme of von Neumann discrimination is 
presented in Fig.~\ref{fig:scheme} and 
can be divided into the following steps:

\begin{enumerate}
	\item Alice and Bob share entanglement state $\ket{\psi_{AB}}$.
	\item Alice performs one of two known measurements on part $A$ of the input 
	state  $\ket{\psi_{AB}}$.
	\item Alice  uses
	the output label $i$ and performs a conditional binary measurement on part 
	$B$.
	\item  Bob measures the part $B$ of the input state  $\ket{\psi_{AB}}$ and 
	makes a decision  basis of the condition binary measurement.
\end{enumerate}  


 

\begin{figure}[h]
	\centering 
	
	\includegraphics[width=0.9\linewidth]{gates1.pdf} 
	\caption{A schematic representation of the setup for distinguishing
		measurements by using entanglement. }
	\label{fig:scheme}
\end{figure}



\subsubsection{Rigetti architecture's limits/ practical solutions}


We may encounter numerous obstacles in the implementation process due to the 
limited architecture of quantum computer. 
Let us take a closer look at the setup for distinguishing von Neumann 
measurements (see Fig.~\ref{fig:diamond}). One of the components of such a 
scheme is to perform the conditional binary measurement. The first limitation 
is lack to possibility of implementation of the conditional binary measurement. 
The only possible measurement performed on quantum architectures is measurement 
in the computational basis. 
A postselection method provides us with a clever going around the obstacle. The 
postselection  scheme (see Fig.~\ref{postselection})  can be described as 
follow. 


\begin{enumerate}
	\item We prepare input state $\ket{\psi_{AB}}$.
	\item One of two measurements is performed on part A.
	\item We prepare optimal measurements $V_j$, which are well known by 
	Holevo--Helstrom theorem and perform them on part $B$.
	\item We get the output label $i$.
	\item We include these cases for which $i = j$. Otherwise, we reject them.
	\item By the use of $V_j$ output, we decide whether the measurement was 
	performed on part $A$.
	\item We count the probability of postselection.
\end{enumerate}

\todo[inline]{PAMIETAJ O DAGGERACH !!!} 

\begin{figure}[h!]
	\includegraphics[scale=1.5]{onequbit.pdf}
	\caption{A schematic representation of the setup for distinguishing
		measurements using postselection.}
	\label{postselection}
\end{figure} 


The schematic representation of theoretical setup is shown in the following 
figure. 

\begin{figure}[h!]
	\centering
	\includegraphics[scale=1.2]{pics/theoretical_scheme}
	\caption{Theoretical discrimination scheme}
	\label{fig:theoretical_scheme}
\end{figure}
\todo[inline]{konretne przyklady zobaczymy potem... }





\subsubsection{Two schemes of practical realization}

\begin{scheme}(By using postsellection)
	
	\begin{figure}[h!]
		\centering 
		\includegraphics[scale=1.2]{pics/postselection} 
		
		\caption{ A schematic representation of the setup for distinguishing
			measurements using postselection. 
		}\label{fig:postsellection}
	\end{figure}
	\begin{enumerate}
		\item We prepare input state $\ket{\psi} = \frac{1}{\sqrt{2} } | \Id_2 \rangle 
		\rangle. $
		\item We prepare one of two unitary channel $\Phi_{U} $ or $\Phi_{\1}$. 
		\item We implement unitary $V_0 $ or $ V_1$.
		\item We prepare the measurement $\Delta$ in standard basis (already exists on 
		Rigetti architecture) on each qubits.
		\item We calculate the probability of correct discrimination in the following 
		way
		\begin{equation}
		p = ...
		\end{equation}
		%\todo[inline]{grubsza kmina, musze Ci to na zywo opowiedziec xd}
	\end{enumerate}
\end{scheme}


\begin{scheme}(By using controlled unitary)
	\begin{figure}[h!]
		\centering 
		\includegraphics[scale=1.2]{pics/controlled_unitary} 
		
		\caption{ A schematic representation of the setup for distinguishing
			measurements using controlled unitary gate. 
		}\label{fig:controlled}
	\end{figure}
	\begin{enumerate}
		\item We prepare input state $\ket{\psi} = \frac{1}{\sqrt{2} } | \Id_2 \rangle 
		\rangle. $
		\item We prepare one of two unitary channel $\Phi_{U} $ or $\Phi_{\1}$. 
		\item We implement controlled unitary $V_0 \oplus V_1$.
		\item We prepare the measurement $\Delta$ in standard basis (already exist on 
		Rigetti architecture) on each qubits.
		\item We calculate the probability of correct discrimination in the following 
		way
		\begin{equation}
		p = \frac{|j=0 \wedge \Phi_{U} | + |j=1 \wedge \Phi_{\1}|}{\text{all 
				cases}}
		\end{equation}
	\end{enumerate}
\end{scheme}




\subsubsection{Theoretical results}



Let $M_{d_1,d_2}$ be the set of all matrices of dimension $d_1 \times d_2$ over
the field $\mathbb{C}$. For  simplicity notation, square matrices will be 
denoted by
$M_d$.  The subset of $M_d$ consisting of Hermitian matrices of dimension $d$ 
will  be  denoted  by $\HH_d$,  while  the  set  of  positive semidefinite 
matrices of dimension $d$ by $\HH_d^+$. The set of quantum states, that is 
positive semidefinite operators having
trace equal to one, will be denoted $\Omega_d$.  The subset of $M_d$ consisting of 
unitary matrices will be denoted by
$\UU_d$, while its subgroup of diagonal unitary operators will be denoted by
$\DD \UU_d$.  We will also need a linear
mapping transforming $M_{d_1}$ into $M_{d_2}$, which will be denoted \begin{equation}
\Phi: M_{d_1 } \rightarrow M_{d_2}.
\end{equation} 
\todo[inline]{nie mowimy channel} 
Finally, we introduce a special subset of all mappings $\Phi$, called quantum channels, which are completely positive
and trace preserving (CPTP).
Mostly, we will use the unitary channels
$\Phi_{U}$ which are defined as $\Phi_U(\cdot) = U \cdot U^\dagger$ for  $U \in 
\UU_d$. A general quantum
measurement, that is a positive operator valued measure (POVM) $\PP$ is a
collection of positive semidefinite operators $\{E_1, \ldots, E_m \}$ called
\emph{effects}, which sum up to identity, \ie $ \, \, \sum_{i=1}^m E_i = \1$. If
all the effects are rank-one projection operators, then such a measurement is
called von Neumann measurement. Every von Neumann measurement can be
parameterized by a unitary matrix and hence we will use the notation $\PP_{U}$
for a von Neumann measurement with effects $\{\proj{u_0}, \ldots, \proj{u_{d-1}}\}$,
where $\ket{u_i}$ is the $i+1$-th column of the unitary matrix $U$. The action of
quantum measurement $\PP_{U}$ on some state $\rho \in \Omega_d$ can be
expressed as the action of a quantum channel
$
\PP_{U} : \rho \rightarrow \sum_{i=0}^{d-1} \bra{u_i} \rho \ket{u_i} \proj{i}.
$ Every von Neumann measurement  $\PP_{U}$ can be also rewritten as $\Delta 
\circ \Phi_{(UE)^\dagger}$, where $\Delta$ denotes the completely dephasing 
channel.  








\subsubsection{Diamond norm}
Let us now consider linear mappings transforming  square matrices into square 
matrices 
that is \ $\Phi: M_{d_1} \to M_{d_2}$. 
 We define its completely bounded trace
norm, also known as a diamond norm, as
\begin{equation}
\|\Phi\|_\diamond = \max_{\|X\|_1=1} \| \left(\Phi \otimes \1\right) (X) \|_1.
\end{equation}


The celebrated result by Helstrom~\cite{helstrom1976quantum} gives the optimal  probability of correct discrimination between two quantum operations, $\TT$  and $\SS$, 
in terms of their distance with the use of the diamond norm
\begin{equation}
p =  \frac12 + \frac14 \| \TT - \SS \|_\diamond.
\end{equation}

In case then 
we consider the discrimination setup between two von Neumann 
measurements $\PP_U$ and $\PP_\Id$, $U\in \UU_d$ we obtain that (from Theorem 1 \cite{puchala2018strategies})
\begin{equation}
\|\PP_U - \PP_\Id\|_\diamond = \min_{E \in \diaguni_d} \|\Phi_{UE} - 
\Phi_\Id\|_\diamond,
\end{equation}
where $\Phi_U$ is unitary channel and 
$\diaguni_d$ be the set of diagonal unitary matrices of dimension $d$. 



\paragraph{Discrimination of unitary channels}

Before we proceed to presenting our main results, we need to briefly discuss the
problem of discrimination of unitary channels.  In order to formulate the condition for perfect
discrimination of unitary channels we introduce the notion of numerical range of
a matrix $A \in M_d$, denoted by $W(A) =\{\bra{x}A\ket{x}: \ket{x} \in 
\mathbb{C}^d, \;
\;\braket{x}{x}=1\}$. The celebrated Hausdorf-T\"oplitz
theorem states that
$W(A)$ is a convex set and therefore $W(A) =\{\tr A \sigma : \sigma \in \Omega_d
\}$. Let us now recall the well-known result for the
distinguishability of unitary channels.
	Let $U \in \UU_d$ and $\Phi_U: \rho \mapsto U \rho U^\dagger$ be a unitary 
	channel. 
	Then 
	\begin{equation}
	\| \Phi_U  - \Phi_{\1} \|_\diamond = 2 \sqrt{1-\nu^2},
	\end{equation}
	where $\nu = \min_{x \in W(U^\dagger)} |x|  $. 

%\begin{proposition}
 Let $U = H \diag(1, e^{i \phi}) H^\dagger$, $\phi \in [0, 2\pi)$ and	let 
 $\Phi_U$ and $\Phi_\Id$ be two unitary channels. The following equation holds 
	\begin{equation}
	\min_{E \in \diaguni_2} \|\Phi_{UE} - 
	\Phi_\Id\|_\diamond = \|\Phi_{U} - 
	\Phi_\Id\|_\diamond
	\end{equation}
%\end{proposition}


%In the general case, the diamond norm of a Hermiticity-preserving maps $\Phi$ 
%can be computed using the Semidefinite Program~\cite{watrous2018theory} and 
%state 
%primal and dual problem in the following form:	

%
%\begin{minipage}{0.495\linewidth}
%	\begin{equation*}
%		\begin{split}
%			\text{\textbf{Primal problem:}} \\
%			\text{maximize:}\quad & \tr(X J(\Phi)) \\[2mm]
%			\text{subject to:}\quad &  \left[ \begin{array}{cc}\Id_{d_2} 
%\otimes \rho & X \\ X^* & \Id_{d_2} \otimes \rho  \end{array} \right] \ge 0 \\
%			& \rho \in \HH_d^+\\ 
%			& X \in M_{d1,d2}.
%		\end{split}
%	\end{equation*}
%\end{minipage}
%\begin{minipage}{0.495\linewidth}
%	\begin{equation*}
%		\begin{split}
%			\text{\textbf{Dual problem:}} \\
%			\text{minimize:}\quad & ||\tr_1(Y)||_\infty \\[2mm]
%			\text{subject to:}\quad & \left[ \begin{array}{cc}Y & -J(\Phi) \\ 
%-J(\Phi) &  Y \end{array} \right] \ge 0 \\
%			& Y \in \HH_{d_1d_2}^+.
%		\end{split}
%	\end{equation*}
%	\vspace{0.5cm}
%\end{minipage} 





\section{Illustrative Examples}

Let us focus on single-qubit von Neumann measurements $\PP_\1$ and $\PP_U$.
Assume that the unitary matrix $U$ is of the form 
\begin{equation}
U = H \diag (1, \ee^{\ii \phi}) H^\dagger,
\end{equation}
where $H$ is the Hadamard matrix of dimension two and $\phi \in [0, 2 \pi)$.
In this section we will compare theoretical probability of correct 
discrimination between these measurements and the realization on real devices.
The theoretical probability can be calculated using Holevo-Helstrom theorem and 
Theorem 1 from~\cite{puchala2018strategies}. 
This probability will be calculated in 
subsection~\ref{sec:example_theoretical_probability}. Later, we will 
discuss the realization of this scheme on real devices in 
subsection~\ref{sec:example_realization}. For such realization, we 
will need the proper input state, (see sec.~\ref{sec:example_discriminator}) 
and final measurement (see sec.~\ref{sec_example_final_measurement}). 

\subsection{Theoretical probability}\label{sec:example_theoretical_probability}
In this section we calculate the theoretical probability of correct discrimination between two von Neumann measurements $\PP_\1$ and $\PP_U$ for $U = H \diag(1, e^{i \phi}) H^\dagger$, where  $\phi \in [0, 2\pi)$. To do that, we will present an auxiliary lemma.   
\begin{lemma}\label{lemma:min-e-optimal}
	Let $U = H \diag(1, e^{i \phi}) H^\dagger$, $\phi \in [0, 2\pi)$ and	let 
	$\Phi_U$ and $\Phi_\Id$ be two unitary channels. The following equation holds 
	\begin{equation}
	\min_{E \in \diaguni_2} \|\Phi_{UE} - 
	\Phi_\Id\|_\diamond = \|\Phi_{U} - 
	\Phi_\Id\|_\diamond,
	\end{equation}
	
\end{lemma}
\begin{proof} Recall that the distance between two unitary channels is given by
	$
	\| \Phi_U  - \Phi_{\1} \|_\diamond = 2 \sqrt{1-\nu^2},
	$
	where $\nu = \min_{x \in W(U^\dagger)} |x|  $ for any $U \in \mathcal{U}_d$. 
	For $U = H 
	\left(\begin{array}{cc}1&0\\0&e^{i \phi}\end{array}\right)  H^\dagger$ the readers briefly observe that  $\nu^2 = 1 - \frac{|1 - e^{i \phi} |^2 }{4}$. So, 
	\begin{equation}
	\|  \Phi_U  - \Phi_{\1} \|_\diamond = | 1 - e^{i \phi} |. 
	\end{equation} 
		It implies that it is enough to prove  \begin{equation}
		\min_{E \in \diaguni_2} \|\Phi_{UE} - 
		\Phi_\Id\|_\diamond  = | 1 - e^{i \phi} |.
		\end{equation}
%		It implies that we can prove  equivalently  the following condition
%		\begin{equation}
%		\max_{E \in \diaguni_2 } \nu_{UE} = \nu_U
%		\end{equation}
This condition is equivalent to 
	\begin{equation}
	\max_{E \in \diaguni_2 } \nu_{E} \coloneqq \nu = \frac{|1 + e^{i \phi} | }{2},
	\end{equation}
	where $\nu_E = \min_{x \in W(U^\dagger E)} |x|. $ 
	We can prove that
	\begin{equation}
	\min_{\ket{x} \in \mathbb{C}^2:   \proj{x} = 1} |\bra{x}U^\dagger\ket{x}| = 
	\min_{\rho \in \Omega_2} |\tr(U^\dagger\rho)|. 
	\end{equation}
	So we obtain \begin{equation}
	\max_{E \in \diaguni_2 } \nu_{E}  = \max_{E \in \diaguni_2 }  \min_{\rho \in 
		\Omega_2} \left| \tr \left( \rho U E \right) \right|
	\end{equation}
	For that, our task is reduce to show that
	\begin{equation}
	\forall E \in \diaguni_2 \,\, | \tr \left(\rho U E\right) | \le \nu. 
	\end{equation}
	Let us define $E = \left(\begin{array}{cc}E_0&0\\0&E_1\end{array}\right)  $ 
	and let us take $\rho = 
	\left(\begin{array}{cc}\frac{1}{2}&0\\0&\frac{1}{2}\end{array}\right) $. 
	From spectral theorem, let us decompose $U$ as
	\begin{equation}
	U= \lambda_1 \ketbra{x_1}{x_1} + \lambda_2 \ketbra{x_1}{x_2}, 
	\end{equation}
	where  for eigenvector $\lambda_1 = e^{i \phi}$ the eigenvector is 
	of the form $\ket{x_1} = 
	\left[\begin{array}{c}\frac{1}{\sqrt{2}}\\-\frac{1}{\sqrt{2}}\end{array}\right]$,
	whereas for  $\lambda_2 = 1$ we have $\ket{x_2} = 
	\left[\begin{array}{c}\frac{1}{\sqrt{2}}\\\frac{1}{\sqrt{2}}\end{array}\right]$.
	Then, we have 
	\begin{equation}
	\begin{split}
	& \forall E \in \diaguni_2 \,\,\, | \tr (\rho U E) | = \frac{1}{2}  \left| \tr \left(
	H \diag(1, e^{i\phi}) H^\dagger E \right) \right| =  \\ &
	\frac{1}{2} \left| \tr\left((  e^{i \phi} \proj{x_1} + 1 \proj{x_2} ) E \right) 
	\right|  = 
	\frac{1}{2} \left| e^{i \phi}  \bra{x_1} E \ket{x_1} + \bra{x_2} E \ket{x_2} 
	\right| = \\& 
	\frac{1}{2} \left| \frac{E_0 + E_1}{2} + e^{i \phi } \frac{E_0+E_1}{2} \right| 
	= 
	\frac{\left| 1+ e^{i \phi } \right|}{2} \left| \frac{E_0 + E_1}{2} \right| \le 
	\nu, 
	\end{split}
	\end{equation}
	which completes the proof.
\end{proof}
\begin{theorem}
The optimal probability of correct discrimination between von Neumann
measurements $\PP_U$ and $\PP_{\Id}$ for $U = H \diag(1, e^{i \phi}) H^\dagger$,
where $\phi \in [0, 2\pi)$ is given by
\begin{equation}
p = \frac{1}{2} + \frac{|1 - e^{i \phi}  |}{4} . 
\end{equation}
\end{theorem}
\begin{proof}
	From Holevo-Helstrom theorem we obtain
	\begin{equation}
	p = \frac{1}{2} + \frac{1}{4} \| \PP_{U} - \PP_{\Id} \|_\diamond.
	\end{equation}
	Whereas, from Theorem 1 in \cite{puchala2018strategies}, we have 
	\begin{equation}
	\|\PP_U - \PP_\Id\|_\diamond = \min_{E \in \diaguni_d} \|\Phi_{UE} - 
	\Phi_\Id\|_\diamond. 
	\end{equation}
	From Lemma~\ref{lemma:min-e-optimal} we show that 
	\begin{equation}
		\min_{E \in \diaguni_2} \|\Phi_{UE} - 
	\Phi_\Id\|_\diamond = \|\Phi_{U} - 
	\Phi_\Id\|_\diamond,
	\end{equation} which is exactly equal to 
	\begin{equation}
	\|\Phi_{U} - 
	\Phi_\Id\|_\diamond = 2\sqrt{1 - \nu^2} = |1-e^{i   \phi }|. 
	\end{equation}
	Finally, we obtain that
	\begin{equation}
	\| \PP_{U} - \PP_{\Id} \|_\diamond =  |1-e^{i   \phi }|.
	\end{equation}
	It implies that
	\begin{equation}
	p  = \frac{1}{2} + \frac{|1-e^{i \phi}|}{4},
	\end{equation} which completes the proof.
\end{proof}





\subsection{Realization}\label{sec:example_realization}
Now we focus on the realization of the discrimination protocol. We will 
discriminate between $\PP_U$ and $\PP_\1$. To do so, we need to know exact 
forms of the input state and final measurement. We will indicate the optimal 
input state in subsection~\ref{sec:example_discriminator} and the final 
measurement in subsection~\ref{sec_example_final_measurement}.


\subsubsection{Discriminator}\label{sec:example_discriminator}

\begin{proposition}
Consider the problem of discrimination between von Neumann measurements $\PP_U$ 
and $\PP_\1$, $U = H\diag(1, e^{i \phi}) H^\dagger $ and $\phi \in [0, 
2\pi)$.  The optimal input state has the form
\begin{equation}
\ket{\psi_{AB}} = \frac{1}{\sqrt{2}} |\Id_2 \rangle \rangle.
\end{equation}
\end{proposition}

%
%
%\begin{theorem}\label{rozrpomiarow}
%Let us $\mathcal{P}_U, \mathcal{P}_\1$ be two von Neumann measurements,  $U = 
%H \diag(1, e^{i \phi}) H^\dagger$, $\phi \in [0, 2\pi)$ such that $0 
%\not\in W(U)$.  If there exists the discriminator $\rho \in \Omega_2$ of the 
%form $\rho = \frac{1}{2}\rho_1 + \frac{1}{2} \rho_2$ such that
%	\begin{enumerate}
%		\item $\rho_1,\rho_2 \in \Omega_2$
%		\item $\Pi_1 \rho_1 \Pi_1 = \rho_1$,
%		\item $\Pi_2 \rho_2 \Pi_2 = \rho_2$,
%		\item  $\mathrm{diag}(\rho_1) = \mathrm{diag}(\rho_2)$,
%	\end{enumerate}
%	where $\Pi_1,\Pi_2 $ are the projectors on the subspaces
%	spanned by the eigenvectors corresponding to $\lambda_1$ and $\lambda_2$ of 
%	$U$. Then  we have 
%	\begin{equation}
%	||\mathcal{P}_U - \mathcal{P}_\Id||_\diamond = 2\sqrt{1- \left|\frac{1+e^{i 
%	\phi}}{2}\right|^2}.
%	\end{equation}
%	%gdzie $\lambda_1, \lambda_n$ jest odpowiednio najmniejszą i największą 
%	%wartością własną $U\in \mathrm{U}(\XX)$.
%\end{theorem}
%\begin{proof} From \ref{th:minE} we have 
%	\begin{equation}
%	\begin{split}
%	||\mathcal{P}_U - \mathcal{P}_{\1}||_\diamond =\min_{E \in \diaguni_2} 
%	||\mathcal{P}_{UE} - \mathcal{P}_{\1}||_\diamond = \min_{E \in 
%	\diaguni_2}|| \Delta \left(\Phi_{E^\dagger U^\dagger} - \Phi_{\1} \right) 
%	||_\diamond \le \\ \le \min_{E \in \diaguni_2} ||\Delta||_\diamond || 
%	\Phi_{E^\dagger U^\dagger} - \Phi_{\1}  ||_\diamond \le \min_{\diaguni_2}|| 
%	\Phi_{E^\dagger U^\dagger} - \Phi_{\1} ||_\diamond.
%	\end{split}
%	\end{equation}
%The diamond norm between two unitary channels can be calculated by 
%	\begin{equation}
%	\min_{E \in \diaguni_2}||\Phi_{E^\dagger U^\dagger} - \Phi_\1||_\diamond  = 
%	\min_{E \in \diaguni_2} 2\sqrt{1 - \min_{\rho \in \Omega_2} |\tr(UE \rho 
%	)|^2}.
%	\end{equation}
%Hence,  it is enough to calculate the formula
%\begin{equation}
%\begin{split}
%\min_{E \in \diaguni_2} 2\sqrt{1 - \min_{\rho \in \Omega_2} |\tr(UE \rho )|^2}.
%\end{split}
%\end{equation}
%Recall, let us note $U$ as
%\begin{equation}
%U= \lambda_1 \ketbra{x_1}{x_1} + \lambda_2 \ketbra{x_1}{x_2}, 
%\end{equation}
%where  for eigenvector $\lambda_1 = e^{i \phi}$ the eigenvector is of 
%the form $\ket{x_1} = 
%\left[\begin{array}{c}\frac{1}{\sqrt{2}}\\-\frac{1}{\sqrt{2}}\end{array}\right]$,
% whereas for  $\lambda_2 = 1$ we have $\ket{x_2} = 
%\left[\begin{array}{c}\frac{1}{\sqrt{2}}\\\frac{1}{\sqrt{2}}\end{array}\right]$.
% 
%Consider two cases:
%
%	$1^\circ$ Consider $\phi = \pi$. Then $U$ is of the form
%	\begin{equation}
%	U = \left(\begin{array}{cc}0&1\\1&0\end{array}\right).
%	\end{equation}
%Then $0 \in W(U)$. 	From Proposition 3 in ~\cite{puchala2018strategies} the 
%measurements $\mathcal{P}_U, \mathcal{P}_\1$ are perfectly distinguishable if 
%and only if there exists $\rho \in \Omega_2$ such that
%	\begin{equation}
%	\mathrm{diag}\left(U^\dagger \rho\right) = 0.
%	\end{equation}
%Hence, for $U $ we have 
%	\begin{equation}
%	0 = \mathrm{diag}\left(U^\dagger \rho\right) = \mathrm{diag} 
%	\left(\left(\begin{array}{cc}0&1\\1&0\end{array}\right)\left(\begin{array}{cc}\rho_{1,1}&\rho_{1,2}\\\rho_{2,1}&\rho_{2,2}\end{array}\right)\right)
%	 =  \mathrm{diag} 
%	\left(\begin{array}{cc}\rho_{2,1}&\rho_{2,2}\\\rho_{1,1}&\rho_{1,2}\end{array}\right).
%	\end{equation}
% It implies that $\rho_{2,1}=\rho_{1,2} = 0$. Therefore $\rho \in \Omega_2$ for 
% which $\mathcal{P}_U, \mathcal{P}_\1$  are perfectly distinguishable is of the 
% form
%	\begin{equation}
%	\rho = \left(\begin{array}{cc}\rho_{1,1}&0\\0&\rho_{2,2}\end{array}\right),
%	\end{equation}
%	where $\rho_{1,1},\rho_{2,2} \ge 0$ and  $\rho_{1,1}+\rho_{2,2}=1$.
%
%	$2^\circ$ Consider  $\phi \not = \pi$. 
%The spectrum of $U$ is of the form $\sigma(U) = \{e^{i \phi},1\}$. 
%Hence $0 \not\in W\left(U\right)$.  Let us note $U$ as 
%	\begin{equation}
%	U= \lambda_1 \ketbra{x_1}{x_1} + \lambda_2 \ketbra{x_1}{x_2}.
%	\end{equation}
%Based on Lemma 5 in~\cite{puchala2018strategies} let us take $\rho_1 = 
%\ketbra{x_1}{x_1}$ and $\rho_2 = \ketbra{x_2}{x_2}$. Obviously,  $\rho_1,\rho_2 
%\in \Omega_2$ and satisfies the requirements of theorem. Hence, the 
%discriminator $\rho \in \Omega_2$ has the form 
%	\begin{equation}
%	\rho = \frac{1}{2} \rho_1 + \frac{1}{2}\rho_2 = 
%	\left(\begin{array}{cc}\frac{1}{2}&0\\0&\frac{1}{2}\end{array}\right)
%	\end{equation}
%%and 
%%	\begin{equation}
%%	||\mathcal{P}_U - \mathcal{P}_{\1}||_\diamond = 2 \sqrt{1 - \left| \frac{1 
%%+ e^{\mathbf{i} \phi}}{2}\right|^2 },
%%	\end{equation}
%
%	
%%	
%%	$1^\circ$ For $E = \1_\XX$ and $\rho  = \frac{1}{2} \rho_1 + \frac{1}{2} 
%%\rho_2$ we have 
%%	\begin{equation}
%%	\tr \left(U\rho \right) = \frac{1}{2} \lambda_1 \tr\left(\rho_1 \Pi_1 
%%\right) + \frac{1}{2} \lambda_2 \tr\left(\rho_2\Pi_2 \right) = 
%%\frac{\lambda_1+\lambda_2}{2}. 
%%	\end{equation}
%%	Therefore, we have
%%	\begin{equation}
%%	2\sqrt{1 -  |\tr(UE \rho )|^2} =  2 \sqrt{1- 
%%\left|\frac{\lambda_1+\lambda_2}{2}\right|^2}.
%%	\end{equation}
%%	$2^\circ$ Let us consider  $E \in \diaguni_2$ given by $E = \sum_{i=1}^2 
%%e_i \ketbra{i}{i}$   and $\rho  = \frac{1}{2} \rho_1 + \frac{1}{2} \rho_2$. 
%%Then we have 
%%	\begin{equation}
%%	\begin{split}
%%	\tr \left(UE\rho \right) &= \frac{1}{2}\lambda_1 \tr \left(\rho_1 \Pi_1 E 
%%\right) + \frac{1}{2}\lambda_2 \tr \left(\rho_2 \Pi_2 E \right) = \\& = 
%%\frac{1}{2} \lambda_1 \sum_{i =1}^2 \bra{i}\rho_1 \ket{i} e_i + \frac{1}{2} 
%%\lambda_2 \sum_{i =1}^2 \bra{i}\rho_2 \ket{i} e_i = \frac{1}{2} 
%%\left(\lambda_1 
%%+ \lambda_2\right)\sum_{i=1}^2 \bra{i}\rho_1 \ket{i} e_i.
%%	\end{split}
%%	\end{equation} 
%%So we have 
%%	\begin{equation}
%%	\left|\tr\left(UE \rho \right) \right| = 
%%\left|\frac{\lambda_1+\lambda_2}{2} \right| \cdot \left| \sum_{i=1}^2 
%%\bra{i}\rho_2 \ket{i} e_i \right| \le \left|\frac{\lambda_1+\lambda_2}{2} 
%%\right|.
%%	\end{equation}
%%	It implies that
%%	\begin{equation}
%%	\sqrt{1 - \left|\frac{\lambda_1+\lambda_2}{2}\right|^2} \le \sqrt{1 -  
%%|\tr(UE \rho)|^2} \le\sqrt{1 - \min_{\rho \in\Omega_2} |\tr(UE \rho )|^2}.
%%	\end{equation}
%%	Hence we obtain
%%	\begin{equation}
%%	\begin{split}
%%	||\mathcal{P}_U - \mathcal{P}_{\1}||_\diamond &=  \max_{\rho \in \Omega_2} 
%%\sum_{i=1}^2  \sqrt{\left(\bra{u_i}\rho\ket{u_i} + \bra{i} \rho \ket{i} 
%%\right)^2 - 4|\bra{i}\rho\ket{u_i}|^2} \le \\& \le 2 \sqrt{1 
%%-\left|\frac{\lambda_1+\lambda_2}{2} \right|^2}
%%	\end{split}
%%	\end{equation
\begin{proof}
	
	\begin{equation}
	\begin{split}
	\| \PP_{U} - \PP_{\Id}  \|_\diamond & =  \max_{\rho \in \Omega_d} \| \left( \Id \otimes \sqrt{\rho} \right) J(\PP_{U} - \PP_{\Id} )  \left( \Id \otimes \sqrt{\rho} \right)  \|_1   \\ & =  \max_{\rho \in \Omega_d} \| \left( \Id \otimes \sqrt{\rho} \right) \sum_{i=0}^{d-1} \proj{i} \otimes \left( \proj{u_i} - \proj{i} \right)^\top  \|_1  \\ & = \max_{\rho \in \Omega_d} \| \sum_{i=0}^{d-1} \proj{i} \otimes \sqrt{\rho}  \left( \proj{u_i} - \proj{i} \right)^\top \sqrt{\rho}  \|_1  \\ & = 
	 \max_{\rho \in \Omega_d} \| \sum_{i=0}^{d-1}\sqrt{\rho}  \left( \proj{u_i} - \proj{i} \right)^\top \sqrt{\rho}  \|_1
	\end{split}
	\end{equation}
	We can prove that for all $\alpha, \beta \ge 0 $, and unit vectors $\ket{x}, \ket{y}$ the following equation holds 
	\begin{equation}
	\| \alpha \proj{x} - \beta\proj{y} \|_1 = \sqrt{(\alpha + \beta)^2 - 4\alpha \beta |\braket{x}{y}|^2}.
	\end{equation}
	By taking $\ket{x} = \frac{\sqrt{\rho} \ket{\bar{u_i}}}{\| \sqrt{\rho} \ket{\bar{u_i}} \|}$ and $ \ket{y} = \frac{\sqrt{\rho} \ket{i}}{\|\sqrt{\rho} \ket{i} \|}$ we obtain 
	\begin{equation}
		\| \PP_{U} - \PP_{\Id}  \|_\diamond  = \max_{\rho \in \Omega_d} \sum_{i=0}^{d-1} \sqrt{\left( \bra{u_i} \rho \ket{u_i} + \bra{i} \rho \ket{i }\right)^2 - 4 | \bra{u_i} \rho \ket{i} |^2}.
	\end{equation}
	\todo[inline]{dokonczyc}
	
	Let us consider  $\rho =   \frac{1}{2}  	\left(\begin{array}{cc}1&0\\0&1\end{array}\right)  $,   we obtain
	\begin{equation}
	\begin{split}
	||\mathcal{P}_U - \mathcal{P}_{\1}||_\diamond &= \sum_{i=0}^1  
	\sqrt{\left(\bra{u_i}\rho\ket{u_i} + \bra{i} \rho \ket{i} \right)^2 - 
		4|\bra{i}\rho\ket{u_i}|^2} = \\&  \sum_{i=0}^1  \sqrt{4 \bra{i}\rho 
		\ket{i}^2 - 4 \left| \frac{\lambda_1 \bra{i}\rho_1 \ket{i} + 
			\lambda_2\bra{i} \rho_2\ket{i}}{2}\right|^2} =\\&  \sum_{i=0}^1  \sqrt{4 
		\bra{i}\rho \ket{i}^2 - 4 \bra{i} \rho \ket{i}^2\left| \frac{\lambda_1 + 
			\lambda_2}{2}\right|^2} = \\&  \sum_{i=0}^1 2 \bra{i} \rho \ket{i} \sqrt{1 
		-\left| \frac{\lambda_1 + \lambda_2}{2}\right|^2 } = \\& 2 \sqrt{1 -\left| 
		\frac{\lambda_1 + \lambda_2}{2}\right|^2 } = 
	2 \sqrt{1 -\left| \frac{1+e^{i \phi}}{2}\right|^2 } = |1-e^{i \phi }|. 
	\end{split}
	\end{equation}
	From the definition of Choi matrix $J(\Phi)$ of $\Phi$ we have
	\begin{equation}
\| (\Id\otimes \sqrt{\rho} ) J(\Phi)  (\Id\otimes \sqrt{\rho} ) \|_1  =  \| (\Phi \otimes \Id) \left(  | \sqrt{\rho}^\top \rangle  \rangle \langle \langle \sqrt{\rho}^\top | \right) \|_1.
	\end{equation}
	So we obtain that \begin{equation}
	\ket{\psi_{AB}} =   | \sqrt{\rho}^\top \rangle  \rangle = \frac{1}{\sqrt{2} } | \Id_2 \rangle \rangle, 
	\end{equation}
	which completes the proof.
\end{proof}


%
%
%\begin{remark}\label{lemma:rho}
%	Let $\rho_{0} = \frac{1}{2} 
%	\left(\begin{array}{cc}1&0\\0&1\end{array}\right)$. Then 
%	\begin{equation}
%	 \max_\rho \left\|(\1\otimes \sqrt{\rho}) J(\PP_U - \PP_{\Id}) 
%	(\1\otimes 
%	\sqrt{\rho})\right\|_1 =    \left\|(\1\otimes \sqrt{\rho_0}) J(\PP_U - 
%	\PP_{\Id}) 
%		(\1\otimes 
%		\sqrt{\rho_0})\right\|_1
%	\end{equation}
%	for  $U = H\diag(1, e^{i \phi}) H^\dagger $, 
%		where $\phi \in [0, 2\pi)$. 
%\end{remark}


\subsubsection{Final measurement}\label{sec_example_final_measurement}
In this subsection we will derive exact form of the final measurement in the
problem of discrimination of von Neumann measurements $\PP_U$ and $\PP_{\Id}$
for $U = H \diag(1, e^{i \phi}) H^\dagger$, where $\phi \in [0, 2\pi)$.


\begin{proposition}
The  final measurement has the form
\begin{equation}
\begin{split}
\mu(0) = \proj{0} \otimes V_0 \proj{0} V_0^\dagger +  \proj{1} \otimes V_1 
\proj{0} V_1^\dagger  \\ 
\mu(1) = \proj{0} \otimes V_0 \proj{1} V_0^\dagger +  \proj{1} \otimes V_1 
\proj{1} V_1^\dagger,
\end{split}
\end{equation}
where for each $\phi \in \mathbb{R}$,  the controlled unitaries $V_0$ and $V_1$ 
have the form
\begin{equation}
V_0 = \left(\begin{array}{cc}i \sin\left( \frac{\pi - \phi}{4} \right)&-i 
\cos\left( \frac{\pi - \phi}{4} \right)\\ \cos\left( \frac{\pi - 
\phi}{4}\right)& \sin\left( \frac{\pi - \phi}{4} \right)\end{array}\right),
\end{equation}
\begin{equation}
V_1 = \left(\begin{array}{cc}-i \cos\left(\frac{\pi - \phi}{4}\right) &i 
\sin\left( \frac{\pi - \phi}{4}\right)\\\sin\left( \frac{\pi - \phi}{4} \right) 
&  \cos\left( \frac{\pi - \phi}{4} \right) \end{array}\right).
\end{equation}
\end{proposition}
\todo[inline]{add proof of the proposition}


From ~\cite{puchala2018strategies}(Proposition 4) and Lemma~\ref{lemma:rho} we 
have 
\begin{equation}
\ket{\psi} = | \sqrt{\rho}^\top \rangle \rangle = \frac{1}{\sqrt{2}} |\Id_2 
\rangle \rangle. 
\end{equation}
From Holevo-Helstrom theorem  we constrain a measurement $\mu$.  
Let us define \begin{equation}
X  = \left( \PP_U \otimes \Id_2 \right)(\proj{\psi}) -  \left( \PP_\Id 
\otimes \Id_2 \right)(\proj{\psi})
\end{equation}
where $\ket{\psi}$ is defined in Remark~\ref{remark:discriminator}. From 
Hahn-Jordan decomposition let \begin{equation}
X = P - Q
\end{equation}
where $P, Q \ge 0 $. Observe, that $P $ and $Q$ are block-diagonal. 
Let us define projectors $\Pi_P$ and $\Pi_Q$ onto  $im(P)$ and $im(Q)$, 
respectively. Then  $\Pi_P$ and $\Pi_Q$ have the following forms
\begin{equation}
\Pi_P = \left(\begin{array}{cc}\proj{x_p}&0\\0&\proj{y_p}\end{array}\right) 
\end{equation}
and 
\begin{equation}
\Pi_Q = \left(\begin{array}{cc}\proj{x_q}&0\\0&\proj{y_q}\end{array}\right) 
\end{equation}
Hence, we define $V_0$ such that
\begin{equation}
\begin{split}
V_0 \ket{x_p} = \ket{0} \\ 
V_0 \ket{x_q} = \ket{1}
\end{split}
\end{equation}
and $V_1$ such that
\begin{equation}
\begin{split}
V_1 \ket{y_p} = \ket{0} \\ 
V_1 \ket{y_q} = \ket{1}
\end{split}
\end{equation}
The explicit form of $V_0$ and $V_1$ we can see in \texttt{one-qubit.nb}.
Finally, we obtain the measurement $\mu$ of the form
\begin{equation}
\begin{split}
\mu(0) = \proj{0} \otimes V_0 \proj{0} V_0^\dagger +  \proj{1} \otimes V_1 
\proj{0} V_1^\dagger  \\ 
\mu(1) = \proj{0} \otimes V_0 \proj{1} V_0^\dagger +  \proj{1} \otimes V_1 
\proj{1} V_1^\dagger  
\end{split}
\end{equation}


%%%%%%%%%%%%%%%%%%%%%%%%%%%%%%%%%5



\paragraph{Construction of $V_0$ and $V_1$}
	For each $\phi \in \mathbb{R}$,  the controlled unitary $V_0$ and $V_1$ 
	have the following form
	\begin{equation}
	V_0 = \left(\begin{array}{cc}i \sin\left( \frac{\pi - \phi}{4} \right)&-i 
	\cos\left( \frac{\pi - \phi}{4} \right)\\ \cos\left( \frac{\pi - 
	\phi}{4}\right)& \sin\left( \frac{\pi - \phi}{4} \right)\end{array}\right),
	\end{equation}
	
		\begin{equation}
	V_1 = \left(\begin{array}{cc}-i \cos\left(\frac{\pi - \phi}{4}\right) &i 
	\sin\left( \frac{\pi - \phi}{4}\right)\\\sin\left( \frac{\pi - \phi}{4} 
	\right) &  \cos\left( \frac{\pi - \phi}{4} \right) \end{array}\right).
	\end{equation}

\todo[inline]{obwody }
\todo[inline]{TU WYNIKI Z RIGETTIEGO}



\subsection{Sample code snippets analysis (optional)}
\todo[inline]{moze tu kod tej postselekcji  i kontrolowanej unitarki ? }
\label{}



%%%%%%%%%%%%%%%%%%%%%%%%%%%%%%%%%%

%Let us consider one-qubit von Neumann measurements $\PP_U$ and $\PP_\Id$. In 
%this simplest case,  we consider two von Neumann measurements $\PP_\1$ and 
%$\PP_U$ for $U = H diag(0,\ee^{\ii \phi}) H$, where $H_d$--Hadamard matrix 
%of dimension $d$, $0 \le \phi < 2\pi$.    By theoretical results, the expected 
%probability $p_{success}$ of correct distinction between two measurements 
%$\PP_\1$ and $\PP_U$  is given by
%\begin{equation}
%p_{success} = \frac{1}{2} + \frac{1}{4} ||\PP_U - \PP_\1 ||_\diamond = 
%\frac{1}{2} + \frac{|1-\ee^{\ii \phi}|}{4}
%\end{equation}
%by the assumption the discriminator is on the form $\ket{\psi} = 
%\frac{1}{\sqrt{2}} ( \ket{00} +  \ket{11} )$. 
%(tu wyniki z rigettiego...)






\section{Impact }


\textbf{This is the main section of the article and the reviewers weight the 
description here appropriately}

Indicate in what way new research questions can be pursued as a result of the 
software (if any).

Indicate in what way, and to what extent, the pursuit of existing research 
questions is improved (if so).

Indicate in what way the software has changed the daily practice of its users 
(if so).

Indicate how widespread the use of the software is within and outside the 
intended user group.

Indicate in what way the software is used in commercial settings and/or how it 
led to the creation of spin-off companies (if so).

\section{Conclusions}
\label{}

Set out the conclusion of this original software publication.

\section{Conflict of Interest}
Please select the appropriate text:

Potential conflict of interest exists:
We wish to draw the attention of the Editor to the following facts, which may 
be considered as potential conflicts of interest, and to significant financial 
contributions to this work. The nature of potential conflict of interest is 
described below: [Describe conflict of interest]

No conflict of interest exists:
We wish to confirm that there are no known conflicts of interest associated 
with this publication and there has been no significant financial support for 
this work that could have influenced its outcome.


\section*{Acknowledgements}

This work was supported by the Foundation for Polish Science (FNP) under grant
number POIR.04.04.00-00-17C1/18-00.


% The Appendices part is started with the command \appendix;
% appendix sections are then done as normal sections
% \appendix

% \section{}
% \label{}

% References:
% If you have bibdatabase file and want bibtex to generate the
% bibitems, please use
%
%  \bibliographystyle{elsarticle-num} 
%  \bibliography{<your bibdatabase>}

% else use the following coding to input the bibitems directly in the
% TeX file.

\begin{thebibliography}{00}

\bibitem{preskill} Preskill, John. "Quantum Computing in the NISQ era and 
beyond." Quantum 2 (2018): 79.
\bibitem{michielsen2017benchmarking} Michielsen, Kristel, et al. "Benchmarking 
gate-based quantum computers." Computer Physics Communications 220 (2017): 
44-55.
\bibitem{zhukov2019quantum} Zhukov, A. A., et al. "Quantum communication 
protocols as a benchmark for programmable quantum computers." Quantum 
Information Processing 18.1 (2019): 1-23.
\bibitem{hamilton2018generative} Hamilton, Kathleen E., Eugene F. Dumitrescu, 
and Raphael C. Pooser. "Generative model benchmarks for superconducting 
qubits." Physical Review A 99.6 (2019): 062323.
\bibitem{benedetti2018generative} Benedetti, Marcello, et al. "A generative 
modeling approach for benchmarking and training shallow quantum circuits." npj 
Quantum Information 5.1 (2019): 1-9.
\bibitem{puchala2018strategies} Puchała, Zbigniew, et al. "Strategies for 
optimal single-shot discrimination of quantum measurements." Physical Review A 
98.4 (2018): 042103.
\end{thebibliography}
\todo[inline]{Please add the reference to the software repository if DOI for 
software  is available. }

\section*{Current executable software version}
\label{}

Ancillary data table required for sub version of the executable software: (x.1, 
x.2 etc.) kindly replace examples in right column with the correct information 
about your executables, and leave the left column as it is.

\begin{table}[!h]
\begin{tabular}{|l|p{6.5cm}|p{6.5cm}|}
\hline
\textbf{Nr.} & \textbf{(Executable) software metadata description} & 
\textbf{Please fill in this column} \\
\hline
S1 & Current software version & For example 1.1, 2.4 etc. \\
\hline
S2 & Permanent link to executables of this version  & For example: 
$https://github.com/combogenomics/$ $DuctApe/releases/tag/DuctApe-0.16.4$ \\
\hline
S3 & Legal Software License & List one of the approved licenses \\
\hline
S4 & Computing platforms/Operating Systems & For example Android, BSD, iOS, 
Linux, OS X, Microsoft Windows, Unix-like , IBM z/OS, distributed/web based 
etc. \\
\hline
S5 & Installation requirements \& dependencies & \\
\hline
S6 & If available, link to user manual - if formally published include a 
reference to the publication in the reference list & For example: 
$http://mozart.github.io/documentation/$ \\
\hline
S7 & Support email for questions & \\
\hline
\end{tabular}
\caption{Software metadata (optional)}
\label{} 
\end{table}

\end{document}
\endinput
%%
%% End of file `SoftwareX_article_template.tex'.
