\documentclass[border={28pt 0pt 1pt 0pt} ]{standalone}

%\usepackage{tikz}
\usepackage{qcircuit}

\newcommand{\Id}{{\rm 1\hspace{-0.9mm}l}}
\newcommand{\PP}{\mathcal{P}}
\newcommand{\ket}[1]{\ensuremath{|#1\rangle}}
\newcommand{\bra}[1]{\ensuremath{\langle#1|}}
\newcommand{\ketbra}[2]{\ensuremath{\ket{#1}\bra{#2}}}
\newcommand{\proj}[1]{\ensuremath{\ketbra{#1}{#1}}}

\begin{document}

\Qcircuit @C=2em @R=1em  {
\lstick{} & \gate{\Phi_U / \Phi_\Id} & 
\multigate{1}{\proj{0}\otimes V_0 + \proj{1} \otimes V_1} & 
\gate{\Delta}  &  \measure{\mathrm{i}}[0]   \cw  
\\
\lstick{} & \qw & \ghost{\proj{0}\otimes V_0 + \proj{1} \otimes V_1} &
\gate{\Delta}  &  \measure{\mathrm{j}}[0] \cw
\inputgrouph{1}{2}{1.4em}{\ket{\psi_{AB}}}{1.5em}  \\ 
}


\end{document}