\documentclass[a4paper,12pt]{article}
\usepackage[utf8]{inputenc}
\usepackage{enumitem}
\usepackage[hyphens]{url}
\newcommand{\1}{{\rm 1\hspace{-0.9mm}l}}

\newenvironment{response}{\vspace{1em}\noindent\textbf{Response}}{\vspace{1em}}

\begin{document}
We would like to thank the reviewers for their comments. Below we provide
responses to their specific remarks.

\section{Reviewer 1}

\begin{enumerate}
  \item  In my opinion, the weakest part of the paper is the lack of a more
    detailed analysis of the pros and cons of the suggested benchmarking scheme
    and its thorough comparison with other existing NISQ benchmarking schemes.
    Thus, I recommend a minor revision where I would ask the authors to better
    describe advantages and disadvantages of the suggested NISQ benchmarking
    scheme with respect to other existing NISQ benchmarking techniques.
    Although the focus of the paper is the PyQBench software, I find it
    important for the reader to have a clear understanding where and how the
    underlying NISQ benchmarking technique and corresponding software are
    superior to other available techniques and software.

    \begin{response}
      We improved the manuscript, which now includes a short section explaining
      the pros and cons of our method in the context of other existing
      benchmarking methods. Due to the length limitations in the journal, we do
      not think this section can become any longer, but we are open to further
      improvement suggestions if the reviewer finds the current analysis
      lacking.
    \end{response}
  \item When discussing the existing quantum computing frameworks, the
    Introduction should also mention the XACC framework which is known for its
    cross-architectural as well as cross-language portability
    (https://iopscience.iop.org/article/10.1088/2058-9565/ab6bf6).
    Additionally, the recent CUDA-Quantum framework backed by NVIDIA
    (https://github.com/NVIDIA/cuda-quantum) is worth mentioning as well.
    \begin{response}
      Both XACC and CUDA Quantum have now been mentioned in the manuscript.
    \end{response}
  \item The cross-entropy benchmarking technique used in the validation of the
    Sycamore-53 QPU in quantum supremacy experiments
    (\url{https://www.nature.com/articles/s41586-019-1666-5}) should also be
    mentioned in the Introduction
    (\url{https://www.nature.com/articles/s41567-018-0124-x}).

    \begin{response}
      We modified manuscript to include mention of cross-entropy benchmarking
      as suggested by the referee.
    \end{response}
  \item The generalization of the suggested benchmarking techniqiue to multiple
    qubits should be described in more detail, explicitly mentioning whether
    the complexity of the derivation and actual benchmarking depends on the
    number of qubits.

    \begin{response}
      We thank the referee for this suggestion. We extended the manuscript with
      a paragraph describing differences between a single--qubit and
      higher-dimensional cases.
    \end{response}
  \item The references in Supplementary Materials are unresolved.
    \begin{response}
      We apologize for this technical error. The references should now be resolved.
    \end{response}
\end{enumerate}
\section{Reviewer 2}
\item Please extend state-of-the-art: The topic is very hot, including ongoing
  standardization activities.
  \begin{response}
    We extended the state-of-the-art section, which now references several more papers.
  \end{response}
  \item Please add more visual material on the SW architecture - UML Diagrams, Component
  Diagrams --> you can put them in the main part of the paper, such
  that the presentation is easier to follow.
  \begin{response}
    We added a diagram summarizing the software architecture.
  \end{response}
\end{document}
